\newpage\subsection*{\Course Worksheet 9 Answers}
\begin{enumerate}
    
    \item 
    \begin{enumerate}
        \item Not possible: if there is a local max at $x=0$, then $f'$ has to change from negative to positive over interval $(-1,1)$, so $f'$ can't be negative over $(-1,1)$. 
        \item Not possible: if there is a local max at $x=0$ and $x=1$ and $f$ is continuous and differentiable everywhere, then there must be a local minimum between $x=0$ and $x=1$. 
    \end{enumerate}
    
    
    
    
    
    \item 
    \begin{enumerate}
        \item $f$ is continuous on $[0,1]$ and differentiable for $(0,1)$, so the function satisfies the conditions of the MVT. 
        \begin{align*}
        	f'(x) 
            &= \ddx (x(1-x))^{1/2}\\
            &= \frac 1 2 (x(1-x))^{-1/2}(1 - 2x) \\
            &= \frac{1 - 2x}{2\sqrt{x(1-x)}}
        \end{align*}
        Now solve:
        \begin{align*}
        	f'(c) &= \frac{f(1)-f(0)}{1-0} \\
            \frac{1 - 2c}{2\sqrt{c(1-c)}} &= \frac 0 1 \\
            1- 2c &= 0\\
            c &= 1/2
        \end{align*}
        
        
        \item $f$ is continuous over $[0,1]$ but is not differentiable over $(0,1)$, so we cannot apply the MVT. 
    \end{enumerate}

	\item
	\begin{enumerate}
	\item $\displaystyle f(x)=\frac{x^3}{3x^2+1}$

    \begin{align*}
    	f'(x) 
        &= \frac{3x^2(3x^2+1)-x^3(6x)}{(3x^2+1)^2} 
        = \frac{9x^9 + 3x^2 - 6x^3}{(3x^2+1)^2}
        = \frac{3x^2(x^2+1)}{(3x^2+1)^2}
    \end{align*}
    The only critical point is $x=0$. And $f'(x) > 0$ for all $x$ except at $x=0$, so $f$ is increasing everywhere, and there are no local extrema and no absolute extrema. 
    
    \item $\displaystyle g(x)=x\ln x$
    \begin{align*}
    	0 &= g'(x) \\
        0 &= \ddx x \ln x \\
        0 &= \ln x + 1 \\
        \ln x &= -1 \\
        x &= e^{-1} 
    \end{align*}
    The only critical point is $x=e^{-1}$. The domain of $g$ is $x>0$.
    \begin{itemize}
    	\item For $x\in (0,e^{-1})$, $g' < 0$, so $g$ is decreasing on this interval.
    	\item For $x\in (e^{-1},\infty)$, $g' > 0$, so $g$ is increasing on this interval.   
    \end{itemize}
    By the first derivative test, there is a local min at $x = e^{-1}$.
    
    
    \item $\displaystyle h(x)=x^{2/3}(x+5)$
	\begin{align*}
    	0 &= h'(x) \\
        0 &= \ddx x^{2/3}(x+5) \\
        0 &= \frac 2 3 x^{-1/3}(x+5) + x^{2/3} \\
        0 &= \frac 1 3 x^{-1/3}(2x + 10 + 3x) \\
        0 &= x^{-1/3}(5x + 10)
    \end{align*}
    The only critical points are $x=-2$ and $x=0$. 
    \begin{itemize}
    	\item For $x\in (-\infty,-2)$, $f' > 0$, so $f$ is increasing on this interval.
    	\item For $x\in (-2,0)$, $f' < 0$, so $f$ is decreasing on this interval.
    	\item For $x\in (0,\infty)$, $f' > 0$, so $f$ is increasing on this interval.
    \end{itemize}
    By the first derivative test, there is a local max at $x=-2$ and a local min $x = 0$.    

\end{enumerate}
\end{enumerate}

















