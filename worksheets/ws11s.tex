\newpage\subsection*{\Course \ Worksheet 11 Answers}
\begin{enumerate}
    
    \item Using Newton's Method:
    \begin{align*} 
    	x_{n+1} &= x_n - \frac{f(x_n)}{f'(x_n)}, \quad n= 0, 1, 2, \ldots \\
        &= x_n - \frac{-3x^3+5x^2-1}{-9x^2+10x}
    \end{align*}
    Starting with $x_0 = 1$, we obtain: 
    \begin{align*} 
    	x_{1} 
        &= x_0 - \frac{-3x_0^3+5x_0^2-1}{-9x_0^2+10x_0} \\
        &= 1 - \frac{-3+5-1}{-9+10} \\
        &= 1 - 1 \\
        &= 0\\
        x_2
        &= x_1 - \frac{-3x_1^3+5x_1^2-1}{-9x_1^2+10x_1} \\
        &= 0 - \frac{0 + 0 -1}{0 + 0}
    \end{align*}    
    It is not possible to calculate $x_2$ using Newton's Method when starting at $x_0=1$, because $x_2$ is undefined.  
    
    \item 
    \begin{align*} 
    	y'(t) &= \frac{3}{t+1} + \tan t \sec^2 t, \quad y(0) = 1 \\
        \Rightarrow y(t) &= 3 \ln(t+1) + \tan^2 t + c \\
        y(0) = 1 &= \ln(0+1) + \tan^2(0) + c = 0 + 0 + c \Rightarrow c = 1 \\
        y(t) &= \ln(t+1) + \tan^2 t + 1
    \end{align*}
    
    \item By inspection: 
    \begin{align*}
    	\int \left( e^{3x} + \sin(2 \pi x) \right) \,dx
        &= \frac 1 3 e^{3x} - \frac{1}{2\pi} \cos(2\pi x) + C \\
    \end{align*}
    Check work: 
    \begin{align*}
    	\ddx \left( \frac 1 3 e^{3x} - \frac{1}{2\pi} \cos(2\pi x) + C \right) 
        &= e^{3x} + \sin(2 \pi x)
    \end{align*}    
   
   \item We need to find a $\delta$ such that
   
   $$|(4x - 5) - 7 | < \epsilon \quad \text{whenver} \quad 0 < |x - 3 | < \delta$$
   
   But: 
   \begin{align*}
   	|(4x - 5) - 7 | = | 4x - 12 | = 4 | x - 3 |
   \end{align*}
   So if we need $|(4x - 5) - 7 | < \epsilon$, then
   \begin{align*}
   	|(4x - 5) - 7 | = 4 | x - 3 | &< \epsilon \\
    | x - 3 | &= \frac 1 4 \epsilon
   \end{align*}
   Choosing $\delta = \epsilon / 4$, then, as required:
      \begin{align*}
   	|(4x - 5) - 7 | = 4 | x - 3 | &< 4\delta = \epsilon 
   \end{align*}
   
   A sketch is not required, but help explain things.
   
\end{enumerate}

    \begin{center}
        
        \vspace{12pt} 
        
        \begin{tikzpicture}[scale=1] 
            \begin{axis}[
            width=6in,
            height=5in,
            %clip=false,
            axis lines=middle,
            xmin=-.1,xmax=3.5,
            ymin=-1, ymax=10.8,
            xtick={1,2,2.75,3,3.25},
            xticklabels={1,2,$3-\delta$,3,$3+\delta$},
            ytick={6,7,8},
            yticklabels={$7-\epsilon$,7,$7+\epsilon$},        
            axis line style={shorten >=-7.5pt, shorten <=-7.5pt},
            xlabel=$x$,
            ylabel=$y(x)$,
            xlabel style={at={(ticklabel* cs:1)},anchor=north west},
            ylabel style={at={(ticklabel* cs:1)},anchor= east},
            grid style={line width=.2pt, draw=gray!30},
            grid=both,
            ]
            \addplot[black,samples=501,domain=1:4,very thick] {4*x - 5};% 
            \addplot[blue,samples=501,domain=0:2.75,thick] {6}; % horiz line
            \addplot[blue,samples=501,domain=0:3.25,thick] {8}; % horiz line
            \addplot +[mark=none,red,thick] coordinates {(3.25, 0) (3.25, 8)}; % vert line
            \addplot +[mark=none,red,thick] coordinates {(2.75, 0) (2.75, 6)}; % vert line
            \end{axis}
        \end{tikzpicture}
        
        \vspace{12pt}
        If $x \in [3-\delta, 3+\delta]$, then $|y-7| < \epsilon$.
    \end{center}