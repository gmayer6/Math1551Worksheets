\newpage\section*{Worksheet 9, \Course, \Semester} 
\noindent \Sections 4.1,4.2,4.3

\subsection*{A Few Definitions and Theorems from Sections 4.1, 4.2, 4.3}

	\begin{itemize}
    	\item \Emph{Local Extrema:} A function has a \Emph{local maximum} at $x = c$ if $f(x) \le f(c)$ for all $x$ in an open interval containing $c$. A function has a \Emph{local minimum} at $x = c$ if $f(x) \ge f(c)$ for all $x$ in an open interval containing $c$. 
        \item \Emph{Critical Points:} An interior point of the domain of $f(x)$ where $f'=0$, or where $f'$ is undefined, is a \Emph{critical point}. 
        \item \Emph{MVT:} If $f(x)$ is a continuous function defined on $[a,b]$, and is differentiable over $(a,b)$. Then there is at at least one point, $c \in (a,b)$, where 
        $$\frac{f(b)-f(a)}{b-a} = f'(c)$$ 
        \item \Emph{Increasing and Decreasing:} If $f'(x) > 0$ on $(a,b)$, then $f$ is \Emph{increasing} on $[a,b]$. If $f'(x) < 0$ on $(a,b)$, then $f$ is \Emph{decreasing} on $[a,b]$.
        \item \Emph{First Derivative Test:} Suppose $f$ has a critical point at $x=c$.
        \begin{itemize}
            \item If $f'(x)$ changes from positive to negative at $c$, then $f$ has a \Emph{local maximum} at $c$.
            \item If $f'(x)$ changes from negative to positive at $c$, then $f$ has a \Emph{local minimum} at $c$.
            \item If $f'(x)$ doesn't change sign from positive to negative at $c$, then $f$ has no local minimum or maximum at $c$.
        \end{itemize}
    \end{itemize}
    
\subsection*{Exercises}

\begin{enumerate}
	\item If possible, sketch a curve or give a formula for a function that has the following properties. If it is not possible to do so, state why. Assume in each case that $f(x)$ is continuous, differentiable, and defined for all values of $x$. 
    
    \begin{enumerate}
        \item $f(x)$ has a local maximum at $x=0$, and $f'(x)<0$ over the interval $(-1,1)$. 
        \item $f(x)$ has a local maxima at $x=0$ and $x=1$, $f(x)$ has no local minima. 
        \item $f(x)$ is odd, and has local maxima at $x=1$ and $x=2$.
    \end{enumerate}
    
	\item Which of the following functions satisfy the conditions of the 
Mean Value Theorem on the interval $[0,1]$? For those that do not, state why. For those that do, identify all values of $c$ so that $f'(c)=\frac{f(b)-f(a)}{b-a}$. 

	\begin{enumerate}
    	\item $\displaystyle f(x)=\sqrt{x(1-x)}$
        \item $\displaystyle f(x)=|x-0.5|$
    \end{enumerate}
    
	\item For each function below: (a) determine the interval(s) on which the 
function is increasing and/or decreasing; (b) Identify the local and 
absolute extreme values (if any) and where they occur. 
	\begin{enumerate} 
    	\item $\displaystyle f(x)=\frac{x^3}{3x^2+1}$
        \item $\displaystyle g(x)=x\ln x$
        \item $\displaystyle h(x)=x^{2/3}(x+5)$
    \end{enumerate} 



\end{enumerate}