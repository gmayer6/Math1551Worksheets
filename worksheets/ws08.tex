\newpage\section*{Worksheet 8, \Course, \Semester} 
\noindent \Sections 3.10, 3.11

\subsection*{Related Rate Problems}

    Solving rate problems tend to involve the following sequence of steps.
    \begin{enumerate}
        \item Read the question.
        \item Draw a diagram.
        \item Introduce variables.
        \item Construct an equation.
        \item Calculate derivative at a point.
        \item Express answer to question using appropriate units. 
    \end{enumerate}
    Please express your final answer with units. 
    
    
\subsection*{Exercises}

\begin{enumerate}
	\item Construct the linearization of $f(x) = \ln(x-1)$ centered at $x=2$ and use it to approximate the value of $f(3)$. Plot your linearization and $f(x)$ on the same graph.
    
	\item An 5ft ladder is leaning against a vertical wall. If the bottom of the ladder is pulled away from the wall at a constant rate of 3 ft/sec, calculate the rate at which the top is sliding down the wall when the bottom is 4ft from the wall. 

	\item The diameter and height of a right circular cylinder are found at a certain instant to be 10cm and 20cm, respectively. If the diameter is increasing at the rate of 1cm/sec, what change in height will keep the volume constant?

	\item A cubical magnet is measured to have side lengths of 4cm within an error range of 0.3cm. Use differentials to find the maximum error in measuring the volume of the magnet. 

	\item A particle moves in a circular orbit $x^2+y^2=1$. As it passes through the point $\left(\frac{1}{2},\frac{\sqrt{3}}{2}\right)$, its $y$-coordinate decreases at the rate of 3 units/sec. At what rate is the $x$-coordinate changing? 

\end{enumerate}