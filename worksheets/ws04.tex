\newpage\section*{Worksheet 4, \Course, \Semester} 
\noindent \Sections 2.1, 2.6

\subsection*{Exercises}

\begin{enumerate}

    \item \TorF
    \begin{enumerate}
        \item If $y(t) \rightarrow 1$ as $t \rightarrow \infty$, then $y$ has the horizontal asymptote $y = 1$, and $y(t)$ is never equal to 1.      
    	\item {\large If $\displaystyle{\lim_{t\rightarrow 2}} \ t^2 \, f(t) = \infty$, then $\displaystyle{\lim_{t\rightarrow 2}} \, f(t) = \infty$ } 
        \item {\large $\displaystyle{\lim_{t\rightarrow \infty}} \left( t - \sqrt{t^2+16}\right) = \displaystyle{\lim_{t\rightarrow \infty}} \left( t - \left( \sqrt{t^2} + \sqrt{16}\right)\right) $ }
        \item {\large $\displaystyle{\lim_{t\rightarrow \infty}} \left( t - \sqrt{t^2+16}\right) = \infty - \infty = 0 $ }        
    \end{enumerate}
    
    \item If possible, sketch the graph of a function that satisfies the following criteria. If it is not possible to do so, state why. It isn't necessary to give a formula for the functions. 
	\begin{enumerate}
    	\item $f(x)$ is continuous, odd, $f(2) < -1$, $\displaystyle{\lim_{x\rightarrow\infty} f(x) = -1}$
    	\item $g(x)$ is continuous, even, $\displaystyle{\lim_{x\rightarrow-\infty} g(x) = -2}$, and $\displaystyle{\lim_{x\rightarrow \infty} g(x) =2}$        
    \end{enumerate}
	\item If possible, evaluate the following limits. If they do not exist, state why.

	\begin{enumerate}\setlength\itemsep{12pt}
    
		\item $\displaystyle{\lim_{x\to 5^-} \left(\frac{3x}{2x-10}\right)}$

		\item $\displaystyle{\lim_{t\to \infty}\ln\left(1+\frac{1}{t}\right)}$
    
		\item $\displaystyle{\lim_{x\to \infty} \left(\frac{2+\sqrt{x}}{2-\sqrt{x}}\right)}$

\end{enumerate}

	\item Identify all asymptotes (horizontal, vertical, oblique) of the function $f(x) = \displaystyle {\frac{x^3-4x^2+3x}{3x^2-6x}}$

	\item The position of an object is given by $y(t) = t^2 + 2t$. 
    \begin{enumerate}
    	\item Give an expression for the average speed of the object over the interval $[1,1+\Delta t]$, where $\Delta t > 0$. 
        \item Use your expression in part (a) to calculate the average speed of the object over the interval $[1,2]$. 
        \item Use your expression in part (a) to calculate the instantaneous speed when $t = 1$. 
    \end{enumerate}
\end{enumerate}    