\newpage\subsection*{Answers}

\SolutionsStatement

\begin{enumerate}
	
    \item 
    \begin{enumerate}
    	\item $\sin(2\pi x)$
        \item It is not possible to construct such a function. If $f(x)$ is even, then $f(x) = f(-x)$, so even functions fail the horizontal line test over the domain $[-1,1]$. 
    \end{enumerate}   
    \item 
    \begin{enumerate} 
    	\item Domain: 
        \begin{align*} 
        25 - x^2 &> 0 \\
        x^2 &< 25 \\
        |x| &< 5
        \end{align*}
        Range: $(-\infty,\ln 25]$.
        \item The domain of $\cos x$ is $x \in \mathbb R$, and its range is $[-1,1]$. The domain of the inverse cosine function is also $[-1,1]$, so the domain of $\cos^{-1} (\cos x)$ is $\mathbb R$. The range is the range of $\cos^{-1}x$, which is $[0,\pi]$.
        \item The domain of $\cos^{-1} x$ is $x \in [-1,1]$, and its range is $[0,\pi]$. The domain of the cosine function is $\mathbb R$, so the domain of $\cos (\cos^{-1} x)$ is $[-1,1]$. The range is the range of $\cos x$, which is $[-1,1]$.
    \end{enumerate}
    \item 
    	\begin{enumerate}
    	\item \begin{align*} 
        	3^{t+1} &= r \\
            (t+1) \ln 3 &= \ln r \\
            t &= \frac{\ln r}{\ln 3} -1 
        \end{align*}
        \item \begin{align*}
        	\ln t + \ln (t+1) &= 1 \\
            \ln (t(t+1)) &= 1 \\
            e^{\ln (t(t+1))} & = e^1\\
            t(t+1) &= e \\
            t^2 + t - e &= 0 \\
            t &= -\frac{1}{2} + \frac 1 2 \sqrt{1+4e}
        \end{align*}
        We ignore the other root of the quadratic, because it is not in the domain of $\ln t$. 
    	\item \begin{align*}
        	\log_2 (\log_3 ( \log_4 t)) &= m \\
            \log_3 ( \log_4 t) &= 2^m \\
            \log_4 t &= 3^{2^m} \\
            t &= {\large 4^{3^{2^m}}}
        \end{align*}
        \item Let $y=2^t$ and multiply by $4y$.
        \begin{align*}
        	2^t+2^{-t} &= \frac{17}{4} \\
            y^2 +1 &= 17y \\
            y^2 -17y +1 &= 0 \\
            (4y-1)(y-4) &= 0 \\
            y &= \frac 1 4, 4 \\
            2^x &= \frac 1 4, 4 \\
            x &= -2, 2
        \end{align*}
    \end{enumerate}
    \item \begin{align*} 
    f(x) &= \frac{e^{2x} -1}{e^{2x} + 1} \\
    x &= \frac{e^{2y} -1}{e^{2y} + 1} \\
    x(e^{2y} +1) &= e^{2y} - 1 \\
   	e^{2y}(x -1) &= -1 - x \\
    e^{2y} & = \frac{x+1}{1-x}\\
    2y &= - \ln \frac{x+1}{1-x} \\
    y &= \frac 1 2 \ln \left( \frac{x+1}{1-x} \right)
    \end{align*}
    \item 
    \begin{align*}
    	4^x &= 2^{-x^2} \\
        x \ln 4 &= -x^2 \ln 2 \\
        x \ln 2^2 &= -x^2 \ln 2 \\        
        2x &= -x^2 \\    
        x^2 -2x &= 0 \\
        x(x-2) &= 0 \\
        x&= 0,2
    \end{align*}
    \item If $C(t)$ is the cost of the car after $t$ years,
    \begin{align*}
    	C(t) &= 20,000(0.8)^t = 20,000(4/5)^t
    \end{align*}
    For part b we solve
    \begin{align*}
    	10,000 &= 20,000(4/5)^t \\
        0.5 &= (4/5)^t \\
        \ln 0.5 &= t \ln (4/5) \\
        t &= \frac{\ln (1/2)}{\ln(4/5)}
    \end{align*}  
\end{enumerate}




